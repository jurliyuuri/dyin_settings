\documentclass[uplatex,a4paper]{jsarticle}

\usepackage[prefernoncjk]{pxcjkcat}
\usepackage[utf8]{inputenc}
\usepackage[T2A]{fontenc}
\usepackage{tipa}
\usepackage{tipx}
\usepackage{amsmath, amssymb}
\usepackage{ascmac}

\begin{document}

{
\title{\Huge 現代ヴィッセンスタンツ語文法}
\author{Skarsna.haltxeafis/渡久地信之}
\date{2017年}
}
\maketitle
\thispagestyle{empty}

% 表紙のページのページ数を消して次からカウントさせるために
% 目次のページの番号をromanにすることで対処している
% 正しいやり方を見付け次第修正が必要かもしれない

\newpage

\pagenumbering{roman}
\tableofcontents

\newpage
\pagenumbering{arabic}

\section{概要}
現代ヴィッセンスタンツ語は,連邦接触前からデュインで話されている言語である.
この言語は古ヴィッセンスタンツ語を基盤としながらも,サラス語や古リパライン語などの影響を受けている.

\section{音韻}
\subsection{基本音韻}
現代ヴィッセンスタンツ語の音韻は次のようになっている.
\begin{table}[htbp]
\begin{center}
 \caption{現代ヴィッセンスタンツ語の音韻表}
 \label{vis_phonology}
 \begin{tabular}[tb]{|c||c|c|c|c|c|c|c|c|c|c|c|c|c|c|} \hline
  転写 & a & b & c & d & dz & e & f & g & h & i & j & k & l & m \\ \hline
  音素 & a & b & s & d & \textyogh & \textepsilon & f & \textscriptg & x & i & j & k & l & m \\ \hline \hline
  転写 & n & o & p & r  & s & t & tz & u & v & w & x & y & z \\ \cline{1-14}
  音素 & n & \textopeno & p & r & z & t & \texttslig & u & v & w & \textesh & y & \texttslig \\ \cline{1-14}
 \end{tabular}
\end{center}
\end{table}

\subsection{異音}
表\ref{vis_phonology}に音韻を示したが,実際には次のような自由異音が認められる.
\begin{itemize}
 \item a: \textscripta
 \item e: e
 \item g: \textgamma, \textinvscr, \textscr
 \item h: h,\textchi
 \item i: \i, \textbari
 \item r: \textfishhookr, \textturnr
 \item u: \textbaru
 \item x: \textctc
\end{itemize}

\section{音節}
ヴィッセンスタンツ語の音節は固有語の場合最大でCCVVCCであるが,
子音及び母音が3つ以上連続する単語は存在しない.
次に実在の単語にて例に示す.
\begin{itemize}
 \item e: …に (V)
 \item nys: 女性 (CVC)
 \item gled: 黒い (CCVC)
 \item bork: 置く (CVCC)
 \item zoine: 住居 (CVV-CV)
\end{itemize}
なお外来語由来の単語についてはこの限りではない.

\section{文法}
現代ヴィッセンスタンツ語は,古ヴィッセンスタンツ語を基盤としているため孤立語的ではあるものの,
他の言語の影響によりやや孤立語の度合いが弱まっている.
また,古ヴィッセンスタンツ語から現代になるまでに複合語などが合成語化しその後語形変化を伴った結果,
屈折的もしくは膠着的に見えるものも存在している.

%しかしながら,屈折語化や膠着語化については通常考えられるよりも緩やかであるとされている.
%これらの要因として考えられていることとして,
%比較的早い時期から識字に関する教育が行われていた説が上げられているが,
%古語から現代語になるまでに語形の変化が見られるという調査もあるため
%現代のところ解明にはいたっていない.

\subsection{語順}
現代ヴィッセンスタンツ語の語順は平叙文の場合にはSVOである.
\begin{enumerate}
 \item cai hatro kacce. 「私は本を読む.」 
\end{enumerate}

疑問文の場合,次のようになる.
\begin{enumerate}
 \item hatro tif kacce? 「あなたは本を読むか?」 \label{enum_interrog1}
 \item tif hatro kacce na? (同上) \label{enum_interrog2}
\end{enumerate}
正書法では項\ref{enum_interrog1}のような記述が望ましいとされているが,
日常会語などでは項\ref{enum_interrog2}のようなサラス語由来の言い回しも使用される.

否定文の場合,次のようなSV ni Oの形式となる.
\begin{enumerate}
 \item cai hatro ni kacce. 「私は本を読まない.」
\end{enumerate}
ただし所有文に関しては次のように2つの形式を取りうる.
\begin{enumerate}
 \item cai vooz ni kacce. 「私は本を持っていない.」 \label{ni_neg}
 \item cai vooz kacce met. 「私は本を持っていない.」\label{met_neg}
\end{enumerate}
項\ref{met_neg}は直訳すると「私は0つの本を持っている.」となる.

この2つの文は,項\ref{ni_neg}が「所有するということをしていない」となり
「その状態にする気がない」の意味合いを持つのに対し
項\ref{met_neg}は単に「現時点の状態ではそうである」の意味合いのみという違いがある.

このような使い分けは,正書法では所有文にのみ現われるものだが口語では存在文にも確認されている.
\begin{enumerate}
 \item fer gel ni e deef. 「彼(女)は家にいない.」\label{ni_neg_gel}
 \item fer gel e deef nict. 「彼(女)は家にいない.」\label{nict_neg_gel}
\end{enumerate}
項\ref{nict_neg_gel}は直訳すると「彼(女)は家ではないところにいる.」となる.

項\ref{ni_neg_gel}は所有文の項\ref{ni_neg}と同様に「その場所に存在するということをしていない」となり
「あえてその場所にいないようにしている」の意味合いを含む.
項\ref{nict_neg_gel}は口語では単に「その場所にはいない」「別のところにいる」の意味合いとなる.

\subsection{代名詞}
現代ヴィッセンスタンツ語の代名詞には性別による違いや活用は存在せず,単複の区別のみが存在する.

単数形に関しては,古ヴィッセンスタンツ語の時代からあまり変化していない.
一方で複数形に関しては,元々単数形にiweを修飾することで表していたものが時代を下るにつれて合成語化し,
その後発音変化等により綴りが変化した結果,単数形が活用しているように見える形となっている.
\begin{screen}
 \begin{center}
  tif iwe $\xrightarrow{\text{合成語化}}$ tifiwe $\xrightarrow{\text{綴りの変化}}$ tifie \\
  (tif「あなた」,iwe 「複数を表す形容詞」)
 \end{center}
\end{screen}
このような語形の変化は特にiweのような付属語的な形容詞において顕著であるため「iwe変化」と呼称されている.

\subsubsection{人称代名詞}
現代ヴィッセンスタンツ語の人称代名詞は表\ref{vic_human_pronoum}のようになっている.
\begin{table}[htbp]
 \caption{現代ヴィッセンスタンツ語の人称代名詞表}
 \label{vic_human_pronoum}
 \begin{center}
  \begin{tabular}{|c||c|c|c|} \hline
   & 一人称 & 二人称 & 三人称 \\ \hline \hline
   単数 & cai & tif & fer \\ \hline
   複数 & caiwe & tifie & feriwe \\ \hline
  \end{tabular}
 \end{center}
\end{table}

\subsubsection{指示代名詞}
現代ヴィッセンスタンツ語の指示代名詞は表\ref{vic_obj_pronoum}のようになっている.
\begin{table}[htbp]
 \caption{現代ヴィッセンスタンツ語の指示代名詞表}
 \label{vic_obj_pronoum}
 \begin{center}
  \begin{tabular}{|c||c|c|c|} \hline
   & 近称 & 遠称 & 示称 \\ \hline \hline
   単数 & hom & dul & xip \\ \hline
   複数 & homie & duliwe & xipie \\ \hline
  \end{tabular}
 \end{center}
\end{table}

ここでいう示称とは「会話の中で初めて登場したもの」や「聞き手の認識の範囲にないと話し手が考えるもの」を表すものである.
通常,示称で表される対象を指し示すなどの明示する動作が伴う.

\subsection{名詞}
ヴィッセンスタンツ語において名詞は孤立語的であるが,
一部の単語においては代名詞で起きたようなiwe変化が見られる.
\begin{itembox}[l]{iwe変化の一例}
 \begin{center}
  nys $\xrightarrow{\text{複数化}}$ nyswe (nys:「女性」) \\
  galpe $\xrightarrow{\text{複数化}}$ galpye (galpe:「子供」)
 \end{center}
\end{itembox}

以下未作成

\subsection{動詞}
\subsection{形容詞}
\subsection{副詞}
\subsection{前置詞}
\subsection{接続詞}
\subsubsection{並列接続詞}
\subsubsection{従属接続詞}
\subsection{関係詞}

\end{document}